\documentclass[]{elsarticle} %review=doublespace preprint=single 5p=2 column
%%% Begin My package additions %%%%%%%%%%%%%%%%%%%
\usepackage[hyphens]{url}

  \journal{Journal for the Society for Putting Things on Top of Other Things} % Sets Journal name


\usepackage{lineno} % add
\providecommand{\tightlist}{%
  \setlength{\itemsep}{0pt}\setlength{\parskip}{0pt}}

\usepackage{graphicx}
\usepackage{booktabs} % book-quality tables
%%%%%%%%%%%%%%%% end my additions to header

\usepackage[T1]{fontenc}
\usepackage{lmodern}
\usepackage{amssymb,amsmath}
\usepackage{ifxetex,ifluatex}
\usepackage{fixltx2e} % provides \textsubscript
% use upquote if available, for straight quotes in verbatim environments
\IfFileExists{upquote.sty}{\usepackage{upquote}}{}
\ifnum 0\ifxetex 1\fi\ifluatex 1\fi=0 % if pdftex
  \usepackage[utf8]{inputenc}
\else % if luatex or xelatex
  \usepackage{fontspec}
  \ifxetex
    \usepackage{xltxtra,xunicode}
  \fi
  \defaultfontfeatures{Mapping=tex-text,Scale=MatchLowercase}
  \newcommand{\euro}{€}
\fi
% use microtype if available
\IfFileExists{microtype.sty}{\usepackage{microtype}}{}
\bibliographystyle{elsarticle-harv}
\ifxetex
  \usepackage[setpagesize=false, % page size defined by xetex
              unicode=false, % unicode breaks when used with xetex
              xetex]{hyperref}
\else
  \usepackage[unicode=true]{hyperref}
\fi
\hypersetup{breaklinks=true,
            bookmarks=true,
            pdfauthor={},
            pdftitle={Probit Analysis of School Closures in Hamilton Ontario},
            colorlinks=false,
            urlcolor=blue,
            linkcolor=magenta,
            pdfborder={0 0 0}}
\urlstyle{same}  % don't use monospace font for urls

\setcounter{secnumdepth}{0}
% Pandoc toggle for numbering sections (defaults to be off)
\setcounter{secnumdepth}{0}
% Pandoc header
\usepackage{booktabs}
\usepackage{longtable}
\usepackage{array}
\usepackage{multirow}
\usepackage{wrapfig}
\usepackage{float}
\usepackage{colortbl}
\usepackage{pdflscape}
\usepackage{tabu}
\usepackage{threeparttable}
\usepackage{threeparttablex}
\usepackage[normalem]{ulem}
\usepackage{makecell}
\usepackage{xcolor}



\begin{document}
\begin{frontmatter}

  \title{Probit Analysis of School Closures in Hamilton Ontario}
    \author[School of Geography and Earth Sciences, McMaster University]{John Merrall\corref{c1}}
   \ead{merralja@mcmaster.ca} 
   \cortext[c1]{Corresponding Author}
      \address[School of Geography and Earth Sciences, McMaster University]{General Sciences Rm. 206, McMaster University, 1280 Main Street West,
Hamilton, Ontario L8S 4K1}
  
  \begin{abstract}
  This is the abstract.
  
  It consists of two paragraphs.
  \end{abstract}
  
 \end{frontmatter}

\emph{Text based on elsarticle sample manuscript, see
\url{http://www.elsevier.com/author-schemas/latex-instructions\#elsarticle}}

\hypertarget{introduction}{%
\section{Introduction}\label{introduction}}

School closures in the city of Hamilton became a contentious issue in
2002, when the Harris government appointed Jim Murray as special
supervisor to oversee a rationalization of the Hamilton-Wentworth
District School Board (HWDSB) in the face of trustee opposition to
closure of under-utilized schools (Honey 2002; Prokaska 2012). The
following years saw a wave of public board school closures and
replacements in Hamilton, aiming to reduce per-pupil education costs in
the face of urban demographic change, and to access new provincial
construction funding. Resistance to the consolidation process eventually
relented, and this accommodation review system has since become
institutionalized at the HWDSB; the Hamilton Catholic board (HWCDSB),
facing similar overcapacity issues, has followed suit.

Given the wave of school reorganizations that occurred over the past
twenty years, it makes sense to look at whether these closures had a
social equity effect: neighbourhoods certainly didn't suddenly find
themselves without any school at all, but distance to schools would have
increased in areas where school closures occurred. Did this increase in
walking distance disproportionately affect the poor?

This paper will examine the likelihood of a HWDSB or HWCDSB school being
closed between the years of 2006 and 2016, dependent on the income and
deprivation characteristics and trends of each census dissemination area
over that same period of time.

\hypertarget{data}{%
\section{Data}\label{data}}

\hypertarget{census-data}{%
\subsection{Census Data}\label{census-data}}

First, a compact set of Hamilton dissemination area level census data
and GIS dissemination area shapefiles for census years 2006 and 2016
were downloaded from CHASS; areas with NA or zero values for Average
After-Tax Income, and NA values for Percent Children 0-5 Low Income,
were dropped from the dataset. Since this paper's analysis is performed
at the 2006 dissemination area level of support, 2016 data was appended
to the 2006 data frame; the 23 dissemination areas of 2006 which were
split by the time of the 2016 census were identified in R, verified
manually in ArcGIS, and 2016 data was then manipulated to append to the
data frames for those subsequently-split 2006 DAs. Then, the 2006 and
2016 data was used to calculate each DA's percent change in average
household after-tax income, and absolute change in population ages 0 to
14.

\hypertarget{flagging-das-for-school-closure}{%
\subsection{Flagging DAs for school
closure}\label{flagging-das-for-school-closure}}

An incomplete set of shapefiles was received from the HWDSB for primary,
middle school, and secondary school catchments from the years 2005 to
2019; this set of files was checked for veracity against the archive of
the HWDSB website available at archive.org, as well as against news
reports of school closures throughout that period, to produce a complete
and checked set of HWDSB school catchment GIS files. For this paper, the
2005-6 primary-school catchment file was then modified to add a flag for
all primary school catchments where the primary school was subsequently
closed by 2016.

A spatial join was then done in ArcGIS between the flagged 2006 HWDSB
elementary school catchment file and the 2006 dissemination area
shapefile, in order to add a flag to each 2006 DA to identify whether
its public school had closed by 2016. The file created was then manually
verified for the condition of each DA, changing the flag values to suit
the rubric in Figure \ref{fig:pubschoolclassrubric}:

\begin{table}

\caption{\label{tab:table1-create}\label{tab:pubschoolclassrubric} Public School Catchment Classification Rubric}
\centering
\resizebox{\linewidth}{!}{
\begin{tabular}[t]{l|l}
\hline
Flag & Meaning\\
\hline
0 & DA is completely outside closed catchment\\
\hline
1 & DA is completely inside closed catchment\\
\hline
2 & DA is about 70-90\% inside closed catchment\\
\hline
3 & DA is about 30-70\% inside closed catchment\\
\hline
4 & DA is only 10-30\% inside closed catchment\\
\hline
9 & DA is outside closed catchment with coincident edge\\
\hline
\end{tabular}}
\end{table}

\hypertarget{exploratory-analysis}{%
\section{Exploratory Analysis}\label{exploratory-analysis}}

\begin{itemize}
\tightlist
\item
  exploratory analysis (density plots)
\end{itemize}

\hypertarget{methodology}{%
\section{Methodology}\label{methodology}}

we will do a bit of exploratory analysis, then a probit

\hypertarget{probit}{%
\section{Probit}\label{probit}}

\begin{itemize}
\item
  do one
\item
  then charts with results
\item
  do another one for Catholic primary schools, chart the results
\end{itemize}

\hypertarget{addendum}{%
\section{Addendum?}\label{addendum}}

Can we also wwork with the data files to do another regression on school
age in 2016 versus neighbourhood characteristics? It looks like poor
children did get given newer schools.

\hypertarget{the-elsevier-article-class}{%
\section{The Elsevier article class}\label{the-elsevier-article-class}}

\hypertarget{installation}{%
\paragraph{Installation}\label{installation}}

If the document class \emph{elsarticle} is not available on your
computer, you can download and install the system package
\emph{texlive-publishers} (Linux) or install the LaTeX package
\emph{elsarticle} using the package manager of your TeX installation,
which is typically TeX Live or MikTeX.

\hypertarget{usage}{%
\paragraph{Usage}\label{usage}}

Once the package is properly installed, you can use the document class
\emph{elsarticle} to create a manuscript. Please make sure that your
manuscript follows the guidelines in the Guide for Authors of the
relevant journal. It is not necessary to typeset your manuscript in
exactly the same way as an article, unless you are submitting to a
camera-ready copy (CRC) journal.

\hypertarget{functionality}{%
\paragraph{Functionality}\label{functionality}}

The Elsevier article class is based on the standard article class and
supports almost all of the functionality of that class. In addition, it
features commands and options to format the

\begin{itemize}
\item
  document style
\item
  baselineskip
\item
  front matter
\item
  keywords and MSC codes
\item
  theorems, definitions and proofs
\item
  lables of enumerations
\item
  citation style and labeling.
\end{itemize}

\hypertarget{front-matter}{%
\section{Front matter}\label{front-matter}}

The author names and affiliations could be formatted in two ways:

\begin{enumerate}
\def\labelenumi{(\arabic{enumi})}
\item
  Group the authors per affiliation.
\item
  Use footnotes to indicate the affiliations.
\end{enumerate}

See the front matter of this document for examples. You are recommended
to conform your choice to the journal you are submitting to.

\hypertarget{bibliography-styles}{%
\section{Bibliography styles}\label{bibliography-styles}}

There are various bibliography styles available. You can select the
style of your choice in the preamble of this document. These styles are
Elsevier styles based on standard styles like Harvard and Vancouver.
Please use BibTeX~to generate your bibliography and include DOIs
whenever available.

Here are two sample references: Feynman and Vernon Jr. (1963; Dirac,
1953).

\hypertarget{references}{%
\section*{References}\label{references}}
\addcontentsline{toc}{section}{References}

\hypertarget{refs}{}
\leavevmode\hypertarget{ref-Dirac1953888}{}%
Dirac, P., 1953. The lorentz transformation and absolute time. Physica
19, 888--896.
doi:\href{https://doi.org/10.1016/S0031-8914(53)80099-6}{10.1016/S0031-8914(53)80099-6}

\leavevmode\hypertarget{ref-Feynman1963118}{}%
Feynman, R., Vernon Jr., F., 1963. The theory of a general quantum
system interacting with a linear dissipative system. Annals of Physics
24, 118--173.
doi:\href{https://doi.org/10.1016/0003-4916(63)90068-X}{10.1016/0003-4916(63)90068-X}


\end{document}


