\documentclass[]{elsarticle} %review=doublespace preprint=single 5p=2 column
%%% Begin My package additions %%%%%%%%%%%%%%%%%%%
\usepackage[hyphens]{url}

  \journal{Journal for Doing Spatial Stuff} % Sets Journal name


\usepackage{lineno} % add
\providecommand{\tightlist}{%
  \setlength{\itemsep}{0pt}\setlength{\parskip}{0pt}}

\usepackage{graphicx}
\usepackage{booktabs} % book-quality tables
%%%%%%%%%%%%%%%% end my additions to header

\usepackage[T1]{fontenc}
\usepackage{lmodern}
\usepackage{amssymb,amsmath}
\usepackage{ifxetex,ifluatex}
\usepackage{fixltx2e} % provides \textsubscript
% use upquote if available, for straight quotes in verbatim environments
\IfFileExists{upquote.sty}{\usepackage{upquote}}{}
\ifnum 0\ifxetex 1\fi\ifluatex 1\fi=0 % if pdftex
  \usepackage[utf8]{inputenc}
\else % if luatex or xelatex
  \usepackage{fontspec}
  \ifxetex
    \usepackage{xltxtra,xunicode}
  \fi
  \defaultfontfeatures{Mapping=tex-text,Scale=MatchLowercase}
  \newcommand{\euro}{€}
\fi
% use microtype if available
\IfFileExists{microtype.sty}{\usepackage{microtype}}{}
\bibliographystyle{elsarticle-harv}
\usepackage{graphicx}
% We will generate all images so they have a width \maxwidth. This means
% that they will get their normal width if they fit onto the page, but
% are scaled down if they would overflow the margins.
\makeatletter
\def\maxwidth{\ifdim\Gin@nat@width>\linewidth\linewidth
\else\Gin@nat@width\fi}
\makeatother
\let\Oldincludegraphics\includegraphics
\renewcommand{\includegraphics}[1]{\Oldincludegraphics[width=\maxwidth]{#1}}
\ifxetex
  \usepackage[setpagesize=false, % page size defined by xetex
              unicode=false, % unicode breaks when used with xetex
              xetex]{hyperref}
\else
  \usepackage[unicode=true]{hyperref}
\fi
\hypersetup{breaklinks=true,
            bookmarks=true,
            pdfauthor={},
            pdftitle={Probit Analysis of School Closures in Hamilton Ontario},
            colorlinks=false,
            urlcolor=blue,
            linkcolor=magenta,
            pdfborder={0 0 0}}
\urlstyle{same}  % don't use monospace font for urls

\setcounter{secnumdepth}{0}
% Pandoc toggle for numbering sections (defaults to be off)
\setcounter{secnumdepth}{0}
% Pandoc header
\usepackage{booktabs}
\usepackage{longtable}
\usepackage{array}
\usepackage{multirow}
\usepackage{wrapfig}
\usepackage{float}
\usepackage{colortbl}
\usepackage{pdflscape}
\usepackage{tabu}
\usepackage{threeparttable}
\usepackage{threeparttablex}
\usepackage[normalem]{ulem}
\usepackage{makecell}
\usepackage{xcolor}



\begin{document}
\begin{frontmatter}

  \title{Probit Analysis of School Closures in Hamilton Ontario}
    \author[School of Geography and Earth Sciences, McMaster University]{John Merrall\corref{c1}}
   \ead{merralja@mcmaster.ca} 
   \cortext[c1]{Corresponding Author}
      \address[School of Geography and Earth Sciences, McMaster University]{General Sciences Rm. 206, McMaster University, 1280 Main Street West,
Hamilton, Ontario L8S 4K1}
  
  \begin{abstract}
  Does the process of school closures and amalgamation negatively affect
  students from poorer socioecoomic backgrounds? This study uses census
  data and historical school catchment maps to study the distributional
  effects of school closures in Hamilton, Ontario from 2006 to 2016.
  \end{abstract}
  
 \end{frontmatter}

\hypertarget{introduction}{%
\section{Introduction}\label{introduction}}

School closures in the city of Hamilton became a contentious issue in
2002, when the Harris government appointed Jim Murray as special
supervisor to oversee a rationalization of the Hamilton-Wentworth
District School Board (HWDSB) in the face of trustee opposition to
closure of under-utilized schools (Honey (2002); Prokaska (2012)). The
following years saw a wave of public board school closures and
replacements in Hamilton, aiming to reduce per-pupil education costs in
the face of urban demographic change, and to access new provincial
construction funding. Resistance to the consolidation process eventually
relented, and this accommodation review system has since become
institutionalized at the HWDSB; the Hamilton Catholic board (HWCDSB),
facing similar overcapacity issues, has followed suit.

Given the wave of school reorganizations that occurred over the past
twenty years, it makes sense to look at whether these closures had a
social equity effect: neighbourhoods certainly didn't suddenly find
themselves without any school at all, but distance to schools would have
increased in areas where school closures occurred. Did this increase in
walking distance disproportionately affect the poor?

This paper will examine the likelihood of a HWDSB or HWCDSB school being
closed between the years of 2006 and 2016, dependent on the income and
deprivation characteristics and trends of each census dissemination area
over that same period of time.

\hypertarget{data}{%
\section{Data}\label{data}}

\hypertarget{census-data}{%
\subsection{Census Data}\label{census-data}}

First, a compact set of Hamilton dissemination area level census data
and GIS dissemination area shapefiles for census years 2006 and 2016
were downloaded from CHASS; areas with NA or zero values for Average
After-Tax Income, and NA values for Percent Children 0-5 Low Income,
were dropped from the dataset. Since this paper's analysis is performed
at the 2006 dissemination area level of support, 2016 data was appended
to the 2006 data frame; the 23 dissemination areas of 2006 which were
split by the time of the 2016 census were identified in R, verified
manually in ArcGIS, and 2016 data was then manipulated to append to the
data frames for those subsequently-split 2006 DAs. Then, the 2006 and
2016 data was used to calculate each DA's percent change in average
household after-tax income, and absolute change in population ages 0 to
14.

\hypertarget{flagging-das-for-school-closure}{%
\subsection{Flagging DAs for school
closure}\label{flagging-das-for-school-closure}}

An incomplete set of shapefiles was received from the HWDSB for primary,
middle school, and secondary school catchments from the years 2005 to
2019; this set of files was checked for veracity against the archive of
the HWDSB website available at archive.org, as well as against news
reports of school closures throughout that period, to produce a complete
and checked set of HWDSB school catchment GIS files. For this paper, the
2005-6 primary-school catchment file was then modified to add a flag for
all primary school catchments where the primary school was subsequently
closed by 2016.

A spatial join was then done in ArcGIS between the flagged 2006 HWDSB
elementary school catchment file and the 2006 dissemination area
shapefile, in order to add a flag to each 2006 DA to identify whether
its public school had closed by 2016. The file created was then manually
verified for the condition of each DA, and a flag was created for each
DA to identify whether it was (mostly or completely) inside the
catchment of a subsequently closed school. The resulting map, showing
wwhich Hamilton DAs experienced a school closure between 2006 and 2016,
is shown in Figure \ref{fig:DA-map}. (Note that the closure of Bell
Stone school in south Glanbrook is not shown, in order to show more
detail for the rest of the city.)

For Catholic school closures, the same work was performed in ArcGIS: a
spatial join was performed between closed Catholic elementary catchments
and DAs, and it was manually verified. In the case of Catholic schools,
however, lack of hard data between 2006 and 2010 meant that some
pre-2010 boundaries had to be assumed. The map showing which Hamilton
DAs experienced a Catholic primary school closure from 2006 to 2016 is
shown in Figure \ref{fig:Cath-map}.

\begin{figure}
\centering
\includegraphics{Probit-Analysis-of-SChool-Closures-in-Hamilton-Ontario_files/figure-latex/print-DA-map-1.pdf}
\caption{\label{fig:DA-map} map of DAs with public primary school
closures}
\end{figure}

\begin{figure}
\centering
\includegraphics{Probit-Analysis-of-SChool-Closures-in-Hamilton-Ontario_files/figure-latex/print-Cath-map-1.pdf}
\caption{\label{fig:Cath-map} Map of DAs with Catholic primary school
closures}
\end{figure}

\hypertarget{exploratory-analysis}{%
\section{Exploratory Analysis}\label{exploratory-analysis}}

An initial investigation can use density plots to determine whether
there is a difference in catchments, between those with a school closure
and those without, for various variables in the census data. In Figure
\ref{fig:densityplots}, for example, it can be seen that for both the
HWDSB and HWCDSB, schools from catchments with a lower average after-tax
household income were more likely to be closed; this, however, could
certainly be due to the difference in incomes across Hamilton, since
areas seeing housing growth (and thus high utilization of schools) are
less likely to see school closures, while older areas with no housing
growth (and thus possible under-utilization of schools that were built
for a larger child population) would be more likely to see schools
close. Contrast this with the density plots for DA delta income in
Figure \ref{fig:deltaincomeplots}, for example: neighbourhoods showing
lower income growth from 2006 to 2016 seem to be marginally \emph{less}
likely to see their school close.

\begin{figure}
\centering
\includegraphics{Probit-Analysis-of-SChool-Closures-in-Hamilton-Ontario_files/figure-latex/fig-right-left-panel-plot-1.pdf}
\caption{\label{fig:densityplots} Income characterists of closed and
non-closed DAs}
\end{figure}

\begin{figure}
\centering
\includegraphics{Probit-Analysis-of-SChool-Closures-in-Hamilton-Ontario_files/figure-latex/fig-right-left-panel-plot2-1.pdf}
\caption{\label{fig:deltaincomeplots} Delta income characterists of
closed and non-closed DAs}
\end{figure}

\hypertarget{probit-analysis}{%
\section{Probit Analysis}\label{probit-analysis}}

A probit of the form

\[
Pr(Y=1 | X) = \Phi(X^T\beta)
\]

can be used to determine whether or not any socioeconomic factors for a
DA are correlated with the binary outcome of school closure. In this
case, Y is a vector of binary (1/0) flags for school closure, \(X\) is a
vector of socioeconomic characteristics for each DA, and \(\beta\) is
the coefficient vector to be solved.

Two separate probit regressions were performed using \emph{glm} in R,
using a limited set of socioeconomic characteristics from the 2006 and
2016 census, to determine whether school closure was more likely to
happen in DAs with certain socioeconomic characteristics. The variables
used in the probit regressions for this paper are shown in Table
\ref{tab:probitvars}; regression results for both Catholic and Public
schools are compared in Table \ref{tab:probitresults}.

\begin{table}

\caption{\label{tab:table1-create}\label{tab:probitvars} Regression variables used}
\centering
\resizebox{\linewidth}{!}{
\begin{tabular}[t]{l|l}
\hline
Variable & Meaning\\
\hline
deltaincome & percent change in DA average aftertax household income, 2006-2016\\
\hline
deltapop0to14 & absolute change in child (0-14) population, 2006-2016\\
\hline
POP0TO142006 & child (0-14) population, 2006\\
\hline
log(AVGAFTERTAXINCHH2006) & natural log of 2006 average after-tax household income\\
\hline
PERCLOWINC0TO52006 & percent of children 0-5 under family low-income threshold, 2006\\
\hline
\end{tabular}}
\end{table}

\begin{table}

\caption{\label{tab:probitresults}\label{tab:probitresults} Regression results: Likelihood of school closure}
\centering
\begin{tabular}[t]{lrlrl}
\toprule
\multicolumn{1}{c}{Variable} & \multicolumn{2}{c}{Public} & \multicolumn{2}{c}{Catholic} \\
\cmidrule(l{3pt}r{3pt}){1-1} \cmidrule(l{3pt}r{3pt}){2-3} \cmidrule(l{3pt}r{3pt}){4-5}
  & $\beta$ & p-val & $\beta$ & p-val\\
\midrule
(Intercept) & 9.260 & < 0.001 & 9.136 & < 0.001\\
deltaincome & -0.966 & 0.017 & -0.193 & 0.541\\
deltapop0to14 & -0.003 & 0.269 & -0.001 & 0.414\\
POP0TO142006 & -0.002 & 0.164 & -0.001 & 0.275\\
log(AVGAFTERTAXINCHH2006) & -0.857 & < 0.001 & -0.923 & < 0.001\\
\addlinespace
PERCLOWINC0TO52006 & -0.001 & 0.755 & 0.003 & 0.258\\
\bottomrule
\multicolumn{5}{l}{\textit{Note: }}\\
\multicolumn{5}{l}{AIC (Public) =  417.67}\\
\multicolumn{5}{l}{AIC (Catholic) =  542.74}\\
\end{tabular}
\end{table}

\hypertarget{discussion}{%
\section{Discussion}\label{discussion}}

As can be seen in Table \ref{tab:probitresults}, there is a significant
negative relationship between 2006 household after-tax income and
likelihood of public or Catholic primary school closure in Hamilton.
This may be simply due to most school closures taking place in the older
lower city where lower income housing is prevalent; newer areas of the
city, which have more expensive houses and thus would require a higher
after-tax income for residents, have seen net new construction of
schools, with only a few school closures taking place south of the
Lincoln Alexander Parkway.

The difference in Catholic and public regressions, for the variable of
delta income, is an interesting result; Figures \ref{fig:DA-map} and
\ref{fig:Cath-map} show a locational difference in lower-city Catholic
and public closures, with closed Catholic schools more likely to be
located south of King Street, with closed public schools more likely to
occur in the north end of the city. The significance of delta income for
the public, and not Catholic, board may thus be an artifact of the
gentrification that took place in the North End over the study period.

Since the public board combined lower-city school closures with
rebuilding of several lower-city schools, a road to further
understanding would be to conduct a regression of school age on the
socioeconomic variables used in this paper, to illustrate whether longer
walking distances to schools caused by HWDSB closures may have been
offset by an improvement in the capital stock of lower-city schools.
Another interesting study would involve seeing whether school age
correlates positively with per-student education costs, to determine if
the economic efficiency of education is affected by building age; this
would require finding per-student education costs for each school.

\hypertarget{references}{%
\section*{References}\label{references}}
\addcontentsline{toc}{section}{References}

\hypertarget{refs}{}
\leavevmode\hypertarget{ref-Honey}{}%
Honey, K., 2002. Supervisor sees surplus for hamilton schools. The
Hamilton Spectator 30 Oct 2002.

\leavevmode\hypertarget{ref-Prokaska}{}%
Prokaska, L., 2012. Spectator's view: School closures painful but
necessary. The Hamilton Spectator 9 May 2012.


\end{document}


